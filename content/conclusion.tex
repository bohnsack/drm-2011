\section{Conclusions and Future Works}
Herein, we started by going over the current generalized state of most cloud systems from an SLA perspective, differentiating between current architectures that incorporate SLA ideas into the design itself with possible futures architectures that incorporate pluggable SLAs with varying indicators and objectives.  We then generalized SLAs into sets of quadruples containing a monitoring function, a set of values defining acceptable ranges returned from the monitoring functions, an evaluation function, and a penalty evaluating function, and demonstrated how this formulation could be used with a specific example.  With this in place, we then demonstrated that the generalized SLA problem is equivalent to $ SAT $, and therefore is NP-Complete.  We finally covered the implications and theoretical limits implied by this NP-Completeness, validating the applicability of this work by designing a realistic control model using these ideas.

This is preliminary work into establishing the theoretical bounds surrounding effective automated control of cloud systems within Internet-scale systems.  Furthermore, the SLA modeled was fairly simple, and only took into account externally-evaluatable metrics in a black-box arrangement.  SLAs can very well outline other operational parameters, like specific data routing, fine-grained usage management, or encryption requirements.  These scenarios are much more difficult to manage than the kinds outlined within this paper.  Likewise, experimental evidence supporting these control ideas will be vital to promoting acceptance and action around these concepts within both cloud service provider systems and user configured applications.